% ------ headers globales -------------
\documentclass[11pt, a4paper, twoside]{article}
\usepackage{header}
\usepackage{config}
\usepackage[lined, boxed, linesnumbered, commentsnumbered]{algorithm2e}
% -------------------------------------
\begin{document}

% -- Carátula --
%\clearpage{\pagestyle{empty}% parametros para la caratula (caratula.sty)

\materia{Sistemas Operativos}
%\submateria{}
\titulo{Trabajo Práctico 1}
\subtitulo{Scheduling}
\fecha{16 de septiembre de 2014}
\integrante{Rodriguez, Pedro}{197/12}{pedro3110.jim@gmail.com}
\integrante{Benegas, Gonzalo Segundo}{958/12}{gsbenegas@gmail.com}
\integrante{Barrios, Leandro Ezequiel}{404/11}{ezequiel.barrios@gmail.com}
%\grupo{Grupo ??}

\maketitle
\cleardoublepage}

%% =====================================================================

Dado un grafo simple $G = (V,E)$ y un entero $k$, se define una $k$-partición de $G$ como
una partición de $V$ en $k$ conjuntos de vértices $V_{1} \dots V_{k}$. Y, dada una función
de peso definida sobre las aristas de $G$, el peso de una $k$-partición es la suma de los pesos
de las aristas intrapartición. Todos los pesos son positivos.

\begin{enumerate}
    \item Se pide desarrollar los siguientes puntos:
    \begin{enumerate}
		\item Relacionar el problema de k-PMP con el problema 3 del TP 1: 
			  \par Primero que nada, es importante aclarar que en problema de las $k$-PMP, siempre
			  vamos a querer que los $k$ conjuntos sean todos no vacíos. Esto es así porque
			  sino podríamos tomar alguno de los conjuntos no vacíos, $V_{i}$ y dividirlo en dos.
			  Como todos los pesos de las aristas son positivos, seguro que el peso total de la nueva
			  partición es menor o igual que el peso total de la partición original (si al dividir a
			  $V_{i}$ me deshize de alguna arista, el peso disminuyó. Sino, el peso permaneció igual
			  que antes). \\
			  El problema 3 del TP 1 consistía en, dada una matriz de peligrosidad, que 
			  nos decía cuál era
			  el peligro de juntar cada par de productos en un mismo camión,  determinar alguna
			  posible asignación de $n$ productos en $k$ camiones de manera tal que en ninguno
			  de los $k$ camiones se sobrepasase cierto umbral $M$ de peligrosidad. \\
			  Podríamos pensar a cada producto como un nodo de un grafo simple $G = (V,E)$, donde 
			  $|V| = n$ y cada par de nodos $u,v$ están conectados por una arista $e = (u,v)$ de
			  peso igual a la peligrosidad que generan los productos $u$ y $v$ al estar en un
			  mismo camión. Viendo al problema de esta manera, podríamos concluir que:
			  \begin{enumerate}
			      \item si pudiéramos encontrar una \textbf{asignación de los $n$ productos a $k$ 
			      camiones} (con la peligrosidad total de cada camión menor o igual que $M$) tal
			      que la suma total de peligrosidad generada por todos los camiones sea $S$, 
			      entonces podemos estar seguros de que la $k$-PMP del grafo $G$ va a ser 
			      \textbf{por lo menos} tan buena como la solución encontrada para el problema
			      de asignar los $n$ productos en $k$ camiones (podemos tomar la partición tal
			      que el $nodo_{i} \in V_{j}$ sii $producto_{i} \in camion_{j}$).
			      Sin embargo, podría haber alguna $k$-PMP en la
			      cual se sobrepase el umbral $M$ para algún $V_{i} \in \{V_{1} \dots V_{k}\}$
			      pero en la que $ \sum\limits_{i=1}^k peligrosidad(V_{i}) < S$. Setear
			      $M = \infty$, no nos serviría, pues el algoritmo del problema 3 del TP 1 
			      devolvería una solución que sólo utiliza un único camión, y no $k$ camiones,
			      que es lo que necesitaríamos.
			      
			      \item si no pudiéramos encontrar ninguna asignación de $n$ productos en $k$ 
			      camiones con la peligrosidad de cada camión menor que $M$, entonces no podemos
			      asegurar nada sobre la existencia de una $k$-PMP de mínimo peso, pues en la
			      $k$-PMP no tenemos ninguna restricción sobre el peso máximo generado por cada
			      elemento de la partición. A menos que $M = \infty$, en cuyo caso podríamos
			      asegurar que el problema de las $k$-PMP tampoco tendría solución.
			      
			      \item si \textbf{sí pudiéramos} encontrar una $k$-PMP de peso total $X \leq kM$
			      tal que, además, $(\forall i=1 \dots k) peso(V_{i}) \leq M$ entonces podríamos
			      utilizar esa misma partición ($nodo_{i} \in V_{j}$ sii $producto_{i} \in 
			      camion_{j}$) como solución a nuestro viejo problema del TP 1. Por el contrario,
			      si existiera $V_{j} \in \{V_{1} \dots V_{k}\}$ tal que $peso(V_{j}) > M $ 
			      entonces no podríamos asegurar nada sobre dicho problema, y deberíamos resolver
			      el problema usando algún otro método.
			      
			      \item si \textbf{no pudiéramos} encontrar una $k$-PMP, de peso
			      total $X \leq kM$, entonces \textbf{seguro}  que en el problema de los camiones
			      no nos alcanzarían $k$ camiones para meter los $n$ productos. Esto es así porque
			      el no existir $k$-PMP de peso menor que $kM$, quiere decir que no hay forma de 
			      distribuir los $n$ productos en $k$ camiones sin sobrepasarme en al menos uno
			      de los camiones del límite $M$ (si la hubiera, hubiéramos podido hallar alguna
			      $k$-PMP tal que en ninguno de los camiones se sobrepase el límite $M$ y nos 
			      serviría como solución para el problema de los camiones).
			      
			  \end{enumerate}
		
		\item Relacionar el problema de $k$-PMP con el problema de coloreo de los vértices
		      de un grafo.
		      Primero que nada, hay que decir que por más que tengamos un método para hallar
		      un $k$-coloreo de un grafo, es muy probable que esto nos nos sirva para hallar una
		      $k$-PMP de dicho grafo puesto que el $k$-coloreo no considera el valor de las 
		      aristas de $G$ sino que simplemente se preocupa por hacer un análisis de 
		      adyacencias y no-adyacencias entre los nodos de dicho grafo.
		      
		      Si somos capaces de determinar una $k$-PMP para cualquier grafo, entonces también
		      somos capaces de determinar la existencia de un $k$-coloreo para cualquier
		      grafo y, si existe, podemos dar explícitamente uno de estos. \\
		      Tomemos el grafo $G$ para el cual queremos hallar un 
		      $k$-coloreo. Para todo par de nodos $f=(u,v)$ que sean adyacentes, modificar el peso
		      de la arista que los une y ponerle un nuevo peso: peso($f$) = $\infty$.
		      Para los pares de nodos $u,v$ que no sean adyacentes, agregar
		      una arista $e=(u,v)$ que los una, y tal que $peso(e) = 1$. Luego, buscamos
		      alguna $k$-PMP en el nuevo grafo $G'$. Si el peso total generado por dicha 
		      partición es menor que infinito, entonces quiere decir que encontramos $k$ 
		      conjuntos de vértices tales que en cada conjunto no hay ningún par de vértices 
		      unidos por una arista de peso infinito (i.e. en cada conjunto, no puede haber
		      dos nodos que fueran adyacentes en $G$, porque sino en $G'$ habría un eje
		      intrapartición que uniría a estos nodos y haría que el peso total del conjunto
		      se vaya a infinito).
		      Entonces, suponiendo que cada $V_{i}$ es no vacío,  
		      si a cada nodo $x \in V_{i}$ asignamos el color $i$, vamos a tener que
		      para todo par de nodos $u,v \in V(G)$ $color(u)=color(v) \rightarrow $ $u$ y $v$ 
		      no son adyacentes en $G$. Entonces, habremos encontrado un $k$-coloreo para el
		      grafo $G$. Si sólo tuviéramos $t < k$ conjuntos no vacíos, sólo 
		      podríamos asegurar que encontramos un $t$-coloreo para el grafo $G$. Pero seguro que
		      esto no sucede, porque como ya explicamos en el punto $(a)$, $k$-PMP va a utilizar las
		      $k$ particiones (si utilizara sólo $t$ particiones, partiendo uno de los conjuntos de 
		      dicha partición sólo se podría mejorar la solución actual). \\
 
		      
		\item Describir situaciones de la vida real que puedan modelarse utilizando k-PMP.
		Algunas situaciones de la vida cotidiana divertidas que pueden ser modeladas utilizando
		$k$-PMP podrían ser:
			  \begin{enumerate}
				\item Quiero distribuir a $n$ alumnos en $k$ aulas de forma que estos se intenten
				copiar lo menos posible en el examen: La probabilidad de que un alumno se intente
				copiar en el examen es mayor cuanto más cómodo se sienta el alumno en el aula. 
				Y un alumno
				se siente más cómodo en el aula cuanto mayor sea la sumatoria del índice de 
				confianza que tiene con el resto de los alumnos de su aula.
				
				\item Un alumno busca distribuir $n$ materias en $k$ cuatrimestres minimizando
				sus horas extra de estudio, si cada par de materias que son cursadas en el mismo
				cuatrimestre hacen que el alumno tenga que estudiar cierta cantidad de horas extra
				para poder aprobarlas.
				\item Se cuenta con $n$ prendas de ropa y con una lavadora. Dependiendo de qué prendas
				de ropa se metan al mismo tiempo, el lavarropas tarda más o menos tiempo en lavarlas. Se busca 
				seleccionar $k$ conjuntos de prendas tales que se minimize el tiempo total que la 
				lavadora esta encendida.
			  \end{enumerate}
    \end{enumerate}
    
    \item Diseñar una \textbf{heurística constructiva golosa} para $k$-PMP y desarrollar los siguientes
    puntos:
    \begin{enumerate}
		\item Explicar detalladamente el algoritmo implementado:
		La heurística que pensamos se basa en algunas observaciones claves: 
		\begin{enumerate}
			\item la primera, es que nuestro
			algoritmo goloso va a tener que ir agregando nodos a cada uno de los conjuntos según algún criterio. Y,
			como lo que se quiere es minimizar la suma total de los costos de los conjuntos, podríamos agregar
			cada nodo al conjunto en el cuál este agrega el menor peso posible. Entonces, según esta observación,
			si llamamos $w_{i}$ al peso total del conjunto $S_{i}$, 
			agregaríamos el nodo $t$ al conjunto $S_{i}$ tal que se minimize $w_{i}$.
			\item la segunda, es que si no tenemos cuidado, haciendo sólo lo que pensamos en el punto anterior,
			en algún punto de la ejecución de nuestro algoritmo, podríamos tener dos conjuntos $S_{i}$ y $S_{j}$ tales
			que $|S_{i}| = 3$ y $|S_{j}| = 20$ y que $t$ agregara el mismo peso $h$ a cualquiera de los conjuntos y el
			algoritmo elijiese meter el nodo $t$ en $S_{i}$.
			Sin embargo, como estamos trabajando con un grafo $G$ aleatorio, 
			en ese caso, lo más probable sería que el nodo $t$ que procesamos 
			agregue menor peso a $S_{i}$ que a $S_{j}$, puesto que $S_{i}$ tiene una menor cantidad de 
			nodos, lo cual hace que sea menos probable que alguno de estos nodos tenga conexión con el nodo $t$. Luego,
			si esto no fuera así (como en el ejemplo que estamos analizando), sino que $t$ agregara el mismo peso a
			ambos conjuntos, uno quisiera agregar $t$ al conjunto $S_{j}$ con el objetivo de juntar en el mismo 
			conjunto a nodos que tienen muchas aristas en común pero que al mismo tiempo estas aristas no suman
			mucho peso y en conjuntos diferentes a nodos que tienen pocas aristas en común pero con mucho peso
			asociado a estas.
			\item otra idea que se podría sumar a esta heurísitica sería recorrer los nodos que vamos agregando a
			cada conjunto no de forma aleatoria sino agregándolos en algún orden. Por ejemplo, agregando primero
			los nodos que generan mayor peso total con todos sus vecinos.
		\end{enumerate}
		A continuación exponemos un pseudocódigo para el algoritmo que implementamos. \\
		\begin{algorithm}[H]
		\SetKwInOut{Input}{input}
		\SetKwInOut{Output}{output}
		  \Input{n: cantidad de nodos, m: cantidad de aristas, k: cantidad de conjuntos, G: grafo}
		  \Output{res: lista de enteros tal que res[i] contiene el índice del conjunto al cuál pertenece el nodo i}
		  vector$<int>$ conjuntos \\
		  vector$<int>$ tam-conjuntos \\
		  \For{cada nodo i = 1 $\dots$ n}{
				\For{cada conjunto j = 1 $\dots$ k}{
					-calcular $m$ = índice del conjunto que, al agregarle el nodo $i$ se le agrega el menor
					valor $v$ posible, donde $v$ es el cociente entre el peso que se le agrega al conjunto
					$m$ al agregarle el nodo $i$ y la cantidad de elementos que tenía el conjunto $m$  \\
					-agregar nodo $i$ al conjunto $m$ (conjuntos[$m$] = $i$) \\
				}
		  }
		  return conjuntos
		\caption{Algoritmo 1}
		\end{algorithm}
		
		\item Calcular el orden de complejidad de peor caso del algoritmo:
		El algoritmo que desarrollamos utiliza dos $for$, que van uno entre $1 \dots n$ y otro entre $1..k$.
		Si la complejidad de lo que hay adentro de estos ciclos es $O(T(n))$, entonces la complejidad total
		del algoritmo sería $O(knT(n))$. \\
		Dentro de estos ciclos lo que hacemos es determinar a qué conjunto $m$ agregar el nodo $i$. Para esto,
		lo que hacemos es recorrer para el nodo actual los $k$ conjuntos ($O(k)$) y en cada pasada calculamos 
		cuál sería el costo que generaría agregar el nodo $i$ al conjunto $k$. Esto lo hacemos de forma lineal 
		en $|V(G)|$, a partir de haber guardado el grafo $G$ en una estructura conveniente. Luego, 
		$T(n) = O(kn)$. \\
		Entonces, la complejidad de nuestro algoritmo en el peor caso es $ O(knT(n)) = O(k^2 n^2)$.
		
		\item Describir instancias de $k$-PMP para las cuales la heurística no proporciona una solución
		óptima. Indicar qué tan mala puede ser la solución respecto de la solución óptima.
		
		
		\item Realizar una experimentación que permita observar la performance del algoritmo en términos
		de tiempo de ejecución en función del tamaño de la entrada.
    \end{enumerate}
\end{enumerate}





%% =====================================================================

\end{document}
